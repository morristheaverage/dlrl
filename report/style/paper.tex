\documentclass{article}

% latex packages, feel free to add more here like 'tikz'
\usepackage{style/conference}
\usepackage{opensans}
\usepackage{graphicx}
\usepackage{biblatex}
\usepackage{fontawesome}
\usepackage[hidelinks]{hyperref}

% to reference, paste the BibTeX obtained from google scholar into references.bib
\addbibresource{references.bib}
\input{style/math_commands.tex}

% replace this title with your own
\title{Winged Horses with an Autoencoder}

\begin{document}
\maketitle
\begin{abstract}
    This paper proposes using an autoencoder to generate images that look like a Pegasus. This abstract should be short and concise, about 8-10 lines long.
\end{abstract}

% this is where the sections begin
\section{Methodology}
The method is to train an autoencoder~\cite{kramer1991nonlinear}, by minimising the squared L2 loss:
\begin{equation}
    \mathcal{L}_{\textrm{AE}} = \mathbb{E}_{\vx\sim p_{\textrm{data}}}[\, \lVert \vx - D(E(\vx)) \rVert^2 ]
\end{equation}
The methodology should be very concise and the mathematical notation should try to follow \href{https://v1.overleaf.com/latex/templates/template-for-iclr-2021-conference-submission/mmpfhsxmqdkp.pdf}{the ICLR conference guidlines \faExternalLink}. If you are not familiar with \LaTeX, you can use \href{https://www.codecogs.com/latex/eqneditor.php}{this online \LaTeX~equation editor \faExternalLink}. You may want to include an architectural diagram:
\begin{center}
    \includegraphics[width=0.5\textwidth]{figures/architecture.pdf}
\end{center}
The architectural diagram above was created using Inkscape and exported to a PDF. This was then uploaded to the figures directory on the left.

\section{Results}
The results look very blurry, where the best batch of images looks like this:
\begin{center}
    \includegraphics[width=0.5\textwidth]{figures/best-batch.png}
\end{center}
From this batch, the most Pegasus-like image (with quite a stretch of the imagination) is:
\begin{center}
    \includegraphics[width=0.075\textwidth]{figures/best-pegasus.png}
\end{center}

\section{Limitations}
It's very difficult to see anything that looks like a Pegasus. In the future, this could be improved by training for more than 10 epochs, although this was not possible due to the time constraints.

\section*{Bonuses}
This submission has a total bonus of -4 marks (a penalty), as it is trained only on CIFAR-10, and the Pegasus has a dark body colour.

% you can have an unlimited number of references (they can go on the 5th page and span many additional pages without any penalty)
\printbibliography
\end{document}